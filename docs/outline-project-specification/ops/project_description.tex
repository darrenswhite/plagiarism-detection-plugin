\section{Project description}
This project involves developing an IntelliJ IDEA plugin. The plugin is to be used primarily by Computer Science students and will aid in detecting plagiarism. It will detect and record file changes in the editor of the IDE and the recorded data will be sent to a server. Files may be changed in multiple ways, some of these include: typing, copy and pasting, automatic code generation, auto complete, refactoring, and external file changes.

The student will have to enter their credentials (e.g e-mail or student number) so that the recorded data can be identified later. The current project will also need to be stored (either with an id or a name). An internal server will be used to store the data for each student for every project (i.e each assignment will have recorded data). Students should be able to submit their recorded data to the server (similar to that of Blackboard or Turnitin).

The stored data will be processed to identify possibly plagiarism cases. Simple techniques may be used like identifying large uses of copy-paste. More complex techniques such as machine learning may also be used for classifying plagiarised data.

Lecturers can later login to a website or application, which could also be running on the same server, and view their students data. After the lecturer is logged in, the processed data will be shown for each student. There should be optional filters to display specific students, modules, or assignments. It may also be possible to flag certain processed data which show signs of possible plagiarism.
