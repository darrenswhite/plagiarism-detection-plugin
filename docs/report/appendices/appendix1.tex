\chapter{Third-Party Code and Libraries}
% If you have made use of any third party code or software libraries, i.e. any code that you have not designed and written yourself, then you must include this appendix. 

% As has been said in lectures, it is acceptable and likely that you will make use of third-party code and software libraries. If third party code or libraries are used, your work will build on that to produce notable new work. The key requirement is that we understand what is your original work and what work is based on that of other people. 

% Therefore, you need to clearly state what you have used and where the original material can be found. Also, if you have made any changes to the original versions, you must explain what you have changed. 

% As an example, you might include a definition such as: 

% Apache POI library - The project has been used to read and write Microsoft Excel files (XLS) as part of the interaction with the client's existing system for processing data. Version 3.10-FINAL was used. The library is open source and it is available from the Apache Software Foundation. The library is released using the Apache License. This library was used without modification.

\begin{itemize}
\item \textbf{IntelliJ IDEA 2018.1 and IntelliJ Plugin SDK Build 173.0} - The plugin was developed for the JetBrains IDE, IntelliJ IDEA. This used the open source IntelliJ Plugin SDK. This SDK is released under the Apache License. This library was used without modification.
\item \textbf{JDK 8} - IntelliJ IDEA is built using Java and so the JDK was also used to develop the plugin. The JDK is licensed under the GNU General Public License. This library was used without modification.
\item \textbf{JUnit 4} - JUnit was used to write tests for the plugin. JUnit is released under the Eclipse Public License. This library was used without modification.
\item \textbf{Python 3} - Python 3 was used to write and develop the back-end server and the post-processor. Python 3 is released under the Python Software Foundation License. This library was used without modification.
\item \textbf{Python 3 Packages} - Multiple packages were also used with Python 3:
  \begin{itemize}
  \item \textbf{Coverage v4.5.1} - Coverage was used for unit tests coverage. Coverage is released under the Apache License. This library was used without modification.
  \item \textbf{Flask v0.12.2} - Flask was used to develop the web service for the back-end server. Flask is licensed under a three clause BSD License. This library was used without modification.
  \item \textbf{Flask-Login v0.4.1} - Flask-Login was used to provide authentication services for the back-end server. Flask-Login is licensed under the MIT License. This library was used without modification.
  \item \textbf{ldap3 v2.4.1} - ldap3 was used to provide a connection the an LDAP server for authentication for the back-end server. ldap3 is licensed under the LGPL v3 license. This library was used without modification.
  \item \textbf{Matplotlib v2.2.2} - Matplotlib was used by the post-processor for displaying the FTS charts. Matplotlib uses a license based on the Python Software Foundation License. This library was used without modification.
  \item \textbf{Mock v2.0.0} Mock was used by the post-processor and server to mock parts of the system for unit tests. Mock is licensed under the BSD License. This library was used without modification.
  \item \textbf{MockupDB v1.3.0} - MockupDB was used to mock the MongoDB client for the server and post-processor. MockupDB is licensed under the Apache License. This library was used without modification.
  \item \textbf{Nose v1.3.7} - Nose was used for unit testing the server and post-processor. Nose is licensed under the GNU Lesser General Public License. This library was used without modification.
  \item \textbf{PyCrypto v2.6.1} - PyCrypto was used by the server and post-processor to encrypt and decrypt the XML data. PyCrypto is licensed under Public Domain License. This library was used without modification.
  \item \textbf{Pygal v2.4.0} - Pygal was used to display the FTS chart on the web application. Pygal is licensed under the GNU Lesser General Public License v3. This library was used without modification.
  \item \textbf{PyMongo v3.6.0} - PyMongo was used by the post-processor and server for connecting to the MongoDB server. PyMongo is licensed under the Apache License. This library was used without modification.
  \end{itemize}
\item \textbf{Docker v18.03.0-ce} - Docker was used to build and deploy containers for the server, post-processor, and database. Docker is licensed under the Apache License. This library was used without modification.
\item \textbf{Docker Compose v1.8.0} - Docker Compose was used to deploy the server, post-processor, and database containers together. Docker Compose is licensed under the Apache License. This library was used without modification.
\item \textbf{MongoDB v3.6} - MongoDB was used to store the submitted XML data. MongoDB is licensed under GNU AGPL v3.0. This library was used without modification.
\end{itemize}
