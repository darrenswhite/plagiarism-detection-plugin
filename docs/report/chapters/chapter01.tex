\chapter{Introduction}
\label{chp:introduction}
% This should be a brief introduction to the plagiarism in academia
\section{Overview}
Plagiarism is becoming popular among students due to ease of access to the internet. Plagiarism.org states that, ``One out of three high school students admitted that they used the Internet to plagiarize an assignment''\cite{PlagiarismorgFacts}. Detecting acts of plagiarism is not a simple task, especially when done manually. Humans are simply not capable of analysing multiple pieces of work and finding similarities. At least not efficiently or quickly. Automatic detection systems already exist but they only analyse the final piece of work (these systems are described more in detail in \autoref{sec:existing-systems}). These systems can be improved by advancing the detection algorithms. This may prove difficult over time as these algorithms become more complex.

A system that analyses work from its inception would allow more targeted and sophisticated detection methods to be performed. These methods could directly identify the plagiarised work, and how it may be transformed to hide any evidence. This system would not act as a sole detection system, but instead alongside existing tools. This would improve the detection rate and provide more evidence for such cases.

\section{Plagiarism and Unacceptable Academic Practice}
Aberystwyth University classifies plagiarism, collusion, and fabrication of evidence of data as acts of UAP (Unacceptable Academic Practice)\cite{AberUniUAP}. The act of plagiarising is to commit fraud by stealing someone's work and not clearly referencing the owner\cite{PlagiarismorgWhat}. Plagiarism has many forms, some of these are described below\cite{AberUniUAP}\cite{PlagiarismorgWhat}\cite{TurnitinPlagiarismSpectrum}\cite{Clough03oldand}.

\begin{itemize}
  \item Copying or cloning another's work without modification (including copying and pasting)
  \item Paraphrasing or modifying another's work without due acknowledgement
  \item Using a quotation with incorrect information about the source
  \item The majority of the work is made up from other sources
  \item Copying an original idea from another persons work
  \item Putting your own name on someone else's work  
\end{itemize}

These forms of plagiarism apply to written work as well as source code. In terms of coding, copying someone's code without modification or acknowledgement is still plagiarism. Automated systems can be put in place to detect plagiarism and UAP.

The main objective of detecting plagiarism, is finding pieces of work which originated from another source. The plagiarised work can often be obfuscated or modified in a way to try and conceal the original work yet keep the same outcome. In source code, this ranges from simple changes to more complex alterations. The program plagiarism spectrum describes the levels of plagiarism that can be done, with level 0 having no modifications and level 6 altering the control flow of the program\cite{Parker1989}. The higher levels make it more difficult to compare pieces of work against each other. Tools that only analyse the final piece of work will struggle with identifying the higher levels in the spectrum. Tracking how a piece of work develops over time will show patterns of obfuscation. As an example, evidence of changing control flow will be tracked. This shows us that the potential of this project could potentially be very successful if developed properly.
