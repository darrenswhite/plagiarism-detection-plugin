\chapter{Iteration 6}
\section{Planning}
This iteration consists of starting this report, researching machine learning algorithms to detect plagiarism, and encrypting the XML file. Below is the list of stories and their assigned story points. The list is displayed in completion order.

\begin{itemize}
\item Begin initial draft of project report: 3 points
\item Research machine learning techniques for plagiarism: 5 points
\item Encrypt stored data: 2 points
\end{itemize}

\section{Implementation}
\subsection{Researching Machine Learning Techniques}
This iteration had less of a focus on development. Instead, this report was started, and more research for machine learning algorithms to detection plagiarism. Due to the time restriction on this project, machine learning techniques were not implemented. However, it was investigated thoroughly to find potential algorithms.

Bandara and Wijayrathna state that most plagiarism detection algorithms are based on structured methods\cite{Bandara2012}. Structured methods are based on the program structure of source code. Another method for plagiarism detection algorithms is attribute counting. This involves extracting various metrics from the source code such as line lengths, and comment frequency. These metrics are used as inputs for the machine learning classifiers. An attribute counting method for detecting plagiarism would work very well with this project. Each change could be used as the metrics. More source code identifications such as automatic code generation, and refactoring, would greatly improve the accuracy. However, the raw data of the changes would be not a viable solution. They would need to be transformed into a more suitable metric for comparison. In the future, it would be possible to implement these machine learning algorithms in the post-processor.

\subsection{Encrypting the XML File}
Previously, it was discussed that a student may attempt to generate a fake XML file. But it was recently discovered that the XML file could just as easily be modified before submission. For example, using \texttt{sed -i 's/source=CLIPBOARD/source=OTHER/' plagiarism\_detection.xml} to replace all clipboard sources with other. This would cause problems with the post-processor identification. The most obvious solution is to encrypt the XML file, or at least to obfuscate the XML data. Previously it was thought that encryption was unnecessary. Due to the \texttt{PersistentStateComponent} class handling the XML serialisation and deserialisation internally, it was impossible to implement encryption at this level (i.e. just encrypting the whole XML file).

A selection of methods were attempted to effectively obfuscate the XML data. The first method used was to encrypt the file paths (the map keys in the XML file). This was successful and one way to expand on this would be to manually obfuscate each value in the map. To encrypt the map values, they would first need to be represented as a String. But not only would they need to encoded as a String, but also decoded back into an Object. Delving into the IntelliJ Plugin SDK, the \texttt{XmlSerializer} class handles serialising objects into an XML structure. Utilising this class, the map was serialised into an XML string. Now each value can be encrypted by performing 128-bit AES encryption on each XML string, then applying base64 encoding. Base64 encoding is used so that the value can be stored as a value string. This string can then be saved as usual and will be used for the persistent state. Upon loading the state, the process is simply reversed. This prevents students from modifying the XML data.

\section{Retrospective}
The velocity for this iteration is 10. The investigation of machine learning techniques has proved useful and these could be implemented in the future if possible. However, due to the time limitation, only basic plagiarism detection will be implemented. The XML file encryption has solved many issues that were present. The tracking of students file changes is now secured and this data cannot be compromised. It also provides tamper prevention, so that the data will not be altered before submission.
