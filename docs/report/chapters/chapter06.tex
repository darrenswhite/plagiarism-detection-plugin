\chapter{Iteration 3}
\section{Planning}
This iteration mainly focused on adding unit tests to the back-end Python server. These stories were initially due for the previous sprint, but were not completed. Below is the list of stories and their assigned story points. The list is displayed in completion order.

\begin{itemize}
\item Show all submissions on staff dashboard: 1 point
\item Add Python nose tests: 3 points
\end{itemize}

\section{Implementation}
\subsection{Displaying Student Submissions for Staff}
Displaying students' submissions on the staff dashboard, was similar to the already existing student dashboard. The only difference was to remove the student uid filter from the database query. Each submission needed to have some form of identification, so each submission was tagged with the user data. The staff dashboard showed the full name of the student in the submissions table.

\subsection{Testing the Server}
Nosetests was used for the unit testing framework in Python 3. Nose extends unittest, which is the included in the Python standard library. Both unittest and nose are similar to JUnit. The mock library was also being used for the unit tests. \textit{Mocking} allows replacing parts of the application, to create specific testing scenarios. Function decorators are used to specify which functions will be mocked and which proxy data will be used.

The first unit tests to be developed were the authentication methods. The LDAP3 authentication system was mocked to "bypass" logging in as a user. This allowed the authentication methods to operate as if the LDAP server responded. Instead, the tests will mock the code to return the required data. If mock was not used, then real credentials would have to be used, and the tests would have to be able to connect to the LDAP server on the Aberystwyth University network. The sign-in test was still incomplete, due to the database methods not being mocked.

MockupDB is a Python 3 package. It is used to create a mock server for testing the MongoDB code. The approach that MockupDB used was different from mocking methods. But the documentation and an online blog was used for reference\cite{DavisMongoDB}. The server code was refactored to account for the mocking of the database. Both the test environment and production environment needed to use the same server instances (Flask application and MongoDB client). If not, then the tests would fail. When the tests were set up, the MongoDB client was replaced with the mocked instance.

The sign-in test sends a POST request to the Flask application (in a new thread). Whenever the database receives a query, the test accepts (or denies) the query, then a response was sent with the appropriate data. In this test case, a query was sent to retrieve the user data (the user that was attempting to sign-in). The test acknowledged the query, and then sent back None. None was sent because the user had not signed in previously. If the user had previously signed in, then the user details would be sent back instead - but that's another test case. Another request was received for inserting data into the database. This is received upon successful authentication. The mocked LDAP server was used for this and simply returned True when authentication was tested. The insertion was acknowledged and the data that was to be inserted is validated. This validation either failed or passed the test.

Once this first sign-in test was in place, new test case scenarios would much be easier to write. Some of the other scenarios are, sign-in after previously signing in, and sign-in with incorrect credentials. These new test scenarios and dashboard tests will be developed in a future sprint.

\section{Retrospective}
The issues with the previous sprint have continued into this iteration unfortunately. The velocity is 4, which also follows the previous iteration. This was due to unforeseen health issues.
